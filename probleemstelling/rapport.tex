\documentclass{article}
\usepackage{cite}
\usepackage{url}
\usepackage{lscape}
\usepackage[final]{pdfpages}
\usepackage{graphicx}
\usepackage{pdflscape}
\usepackage{rotating}
\usepackage{booktabs}
\usepackage{array}
\usepackage[dutch]{babel}

\begin{document}

\title{Optimized genome sequencing: onderzoeksprobleem}

\author{Brecht Gossel\'e}
\maketitle


\section{Probleemschets}

\subsection{Algemeen}

Wetenschappelijke vooruitgang heeft ervoor gezorgd dat de kost om genomen te sequencen het afgelopen decennium exponentieel gedaald is, sinds 2008 zelfs aan een hogere snelheid dan de evolutie volgens de wet van Moore \cite{wetterstrand_sequencing_cost}. In allerhande soorten biologisch, medisch en pharmaceutisch onderzoek worden dan ook genomen van meer en meer organismen gesequenced en dit genereert enorme hoeveelheden data. Ter illustratie: de \textit{whole genome sequencing pipeline} van het Broad Institute \cite{broad_institute}, een referentie in het veld, genereert bij het sequencen van 1 volledig menselijk genoom in de orde van 3TB aan tussentijdse data. Het eindresultaat is 50 GB gecomprimeerde data voor 1 menselijk genoom bij 50x \textit{coverage} (een maat voor de resolutie \cite{coverage_definition}). Naarmate genomen van miljoenen mensen en andere levende wezens geanalyseerd en opgeslagen worden, vereist deze evolutie steeds betere schaalbaarheid, responstijd, en parallellisering voor de opslag en verwerking van deze data.\\
Een logische stap is om deze problemen aan te pakken met grote verdeelde computersystemen, zogenaamde \textit{high performance computing systems} of \textit{supercomputers}. Het Exascience Life Lab van imec, Intel, Janssen Pharmaceutica en de 5 Vlaamse universiteiten verricht specifiek onderzoek naar de toepassing van supercomputers om het genoomsequencingproces te versnellen \cite{lifelab_bwa}\cite{exascience_life_lab}.\\
De snel toegenomen populariteit van webdiensten als sociale netwerken zadelde webbedrijven op met een gelijkaardige explosie aan data. Om deze zogenaamde Big Data \cite{mashey1997big} adequaat te beheren, volstaan traditionele relationele DBMS niet meer. Daarom hebben grote webbedrijven zoals Google, Amazon en Facebook nieuwe opslagtechnieken ontwikkeld die voldoen aan de vereisten qua incrementele schaalbaarheid, lage responstijden en hoge beschikbaarheid \cite{baker2011megastore}. Dit heeft vele zogenaamde NoSQL ('Not only SQL') databases voortgebracht, die het rigide relationele datamodel inruilen voor betere schaalbaarheid en gemakkelijkere distributie van de data. Daarnaast is er ook de recentere opkomst van NewSQL-systemen: deze trachten de schaalbaarheid, distributie en fouttolerantie van NoSQL-systemen te combineren met het relationele datamodel en de bijhorende SQL-query interface en sterke garanties op gebied van concurrency en consistentie.

\subsection{GEMINI}

GEMINI (van GEnome MINIng) is een tool voor flexibele analyse van genetische variaties in grote datasets van genomen, ontwikkeld aan de University of Virginia \cite{10.1371/journal.pcbi.1003153}. GEMINI vertrekt daarbij van VCF (Variant Call Format) bestanden, laadt deze in in een database, en biedt dan de mogelijkheid queries uit te voeren op deze databank. De huidige operationele versie van GEMINI draait op de eenvoudige SQL-databank SQLite, maar deze laat op gebied van schaalbaarheid en performantie te wensen over. Een eerste oplossing om de performantie te verhogen was om over te schakelen op PostgreSQL, en hiervan is intussen al met succes een eerste versie ge\"implementeerd. Het resultaat is een verhoging van de querysnelheden van 5 tot 20x, zonder verregaande optimalisaties. De verwachtingen zijn echter ook hier dat PostgreSQL op langere termijn niet zal kunnen schalen naar datasets van (honderd)duizenden genomen.\\
GEMINI bewaart de data over genetische varianten in enkele zeer grote tabellen. Een eerste vereiste voor een database is dus om deze tabulaire data goed te kunnen voorstellen en beheren, bij voorkeur d.m.v. automatische verspreiding over verschillende nodes in een cluster. Bovendien voert GEMINI na het inladen van de VCF bestanden enkel nog read-queries uit, dus zijn de belangrijkste verdere vereisten voor een database hoge read-throughput, goede query-mogelijkheden en indexeringsmechanismes. Omdat GEMINI in Python ge\"implementeerd is, is een Python-API ook nuttig. 

\section{Doelstellingen}

Het doel van mijn masterproef is te onderzoeken hoe en welke NoSQL- of NewSQL-systemen kunnen helpen om de genoomsequencing pipeline te versnellen. Daarbij komt bovendien kijken dat verschillende delen van de pipeline verschillende eigenschappen en dus uiteenlopende vereisten hebben.\\
Een eerste stap in mijn onderzoek is dan ook een literatuurstudie om het probleemdomein te leren kennen, verschillende NoSQL- en NewSQL-systemen met elkaar te vergelijken en in te schatten welke het meest geschikt zouden zijn voor welk facet van het sequencingproces.\\
De volgende stap is om met de opgedane kennis te besluiten welke data store het meest geschikt is om GEMINI te verbeteren, dit te implementeren en door testen te verifiëren of de performantie van GEMINI in combinatie met een NoSQL- of NewSQL-systeem aan de verwachtingen voldoet, dit te vergelijken met de bestaande implementaties en te onderzoeken vanaf welke grootte van dataset het de moeite loont om van PostgreSQL over te schakelen op een NoSQL- of NewSQL-oplossing.

\section{Aanpak}

\subsection{Literatuurstudie}

De literatuurstudie heb ik intussen afgerond; de resultaten hiervan heb ik beschreven in een eerste wetenschappelijke paper \cite{gossele_survey}. Ik heb enkele van de populairste datastores in 3 relevante categorieën bestudeerd: de NoSQL document stores MongoDB en Couchbase Server, de NoSQL columnar stores Cassandra en HBase en de NewSQL-systemen Cloudera Impala en VoltDB. Als maat voor de populariteit heb ik de website DB-Engine Ranking \cite{db_engine_rank} gevolgd, die de populariteit van databases inschat op basis van een aantal parameters zoals aantal vermeldingen op het web, zoekfrequentie in verschillende zoekmachines en jobaanbiedingen die de databases vermelden. Deze 6 systemen heb ik vervolgens vergeleken op een aantal eigenschappen die van toepassing zijn voor high performance computing, zoals indexeringsmechanismes, client interfaces naar de data, distributiestrategie, concurrency controle en consistentiemodellen. De resultaten van de vergelijkende studie zijn samengevat in de onderaan bijgevoegde tabel.

\subsection{Implementatie}

De eerste prioriteit in de implementatie is kiezen welke van de bestudeerde databases het meest geschikt is voor GEMINI. Een voor de hand liggende keuze is Cassandra: dit is een matuur product dat ervoor bekend staat te schalen tot enorme datasets, biedt de nodige query- en indexeringsmechanismes, een Python API en bovendien mapt de huidige data layout van GEMINI goed op het columnaire model van Cassandra.\\
De volgende stap is om dit daadwerkelijk te implementeren: uitdokteren hoe de data uit GEMINI exact in Cassandra voor te stellen en vervolgens de connectie tussen GEMINI en Cassandra te programmeren. Dit is vooral het inladen van de data uit VCF files, en het omzetten van queries uit de GEMINI command line interface naar geldige queries in de query taal van Cassandra (CQL of Cassandra Query Language, een sterk op SQL lijkende taal). GEMINI heeft reeds een uitgebreide suite unit-tests die de implementatie zullen vergemakkelijken.
Eenmaal de implementatie werkt en een Cassandra cluster met een testdataset aan genoom informatie operationeel is, kan de combinatie GEMINI-Cassandra gebenchmarkt worden door de prestaties op een aantal veelgebruikte queries te meten.\\\\
Als er tijd overblijft, kan ik nog een tweede database uitproberen en implementeren. Couchbase Server is een andere interessante optie: deze staat dankzij zijn memcached-protocol bekend om z'n lage latency, biedt ook query- en indexing-mechanismes aan (maar iets minder uitgebreid dan Cassandra), heeft eveneens een Python API en het datamodel van GEMINI mapt vrij goed op de JSON-documenten architectuur van Couchbase.

\begin{landscape}
\includepdf[angle=90]{NoSQL_table.pdf}
\end{landscape}

\bibliography{biblio}{}
\bibliographystyle{plain}
\end{document}
