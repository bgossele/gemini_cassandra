\documentclass{article}
\usepackage{cite}

\begin{document}
\subsection{Cassandra}

Apache Cassandra, originally developped at Facebook but later open-sourced, combines the data-model of Google's BigTable system with the architecture and distribution strategy of Amazon's DynamoDB. It is intended for flexible, highly-available storage of very large datasets, running on cheap commodity hardware and offering high write throughput while not sacrificing read efficiency \cite{lakshman2010cassandra}.\\\\

%datamodel en CQL

%distribution + concurrency + CAP dink
In order to be able to scale linearly in the number of nodes to very large datasets, Cassandra operates in a fully masterless fashion. In terms of the CAP-theorem, it focusses on availability and partition-tolerance, rather than immediate consistency (though the user has control over the level of consistency, as will be explained). %TODO concurrency (quorum, CAS, row-atomicity) ; tunable consistency; failure detection/handling

%extra's (Spark, Solr)

%afronden

\bibliography{biblio}{}
\bibliographystyle{plain}
\end{document}