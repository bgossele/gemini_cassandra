\chapter{Achtergrond databanken}
\label{cha2}

Sinds de jaren '70 zijn zogenaamde \textit{relational database management systems} (kortweg RDBMS) de voornaamste technologie voor de grootschalige opslag van gegevens. Ze zijn gestoeld op 2 belangrijke principes, namelijk het relationele datamodel \cite{codd1970relational} en de gestructureerde querytaal SEQUEL, beter gekend als SQL \cite{chamberlin1974sequel}. De architectuur van vele RDBMS is nog steeds gebaseerd op de eerste implementatie van een dergelijk systeem, namelijk het IBM onderzoeksproject System R \cite{blasgen1981system}, ook uit halverwege de jaren '70. System R is uiteraard ontworpen voor destijds relevante hardwarekarakteristieken en productvereisten: business data processing via een command line interface, en dit op computersystemen met trage processoren, kleine werk- en schijfgeheugens maar relatief grote bandbreedte tussen de schijfopslag en het werkgeheugen \cite{Stonebraker:2007:EAE:1325851.1325981}. Dit leidde tot een aantal architecturale features die nog steeds terug te vinden zijn in hedendaagse RDBMS:
\begin{itemize}
\item Disk-geori\"enteerde opslag- en indexstructuren
\item Multithreading om latency te verbergen
\item Concurrency-controlemechanismen op basis van locking
\item Log-gebaseerd herstel van fouten
\end{itemize} 
Ondanks gigantische technologische vooruitgang op gebied van hardware en sterk gediversifieerde gebruiksscenario's, is er sinds hun ontstaan 40 jaar geleden weinig drastisch veranderd aan het concept van de RDBMS en zijn deze systemen de werkpaarden van de industrie geworden op het vlak van dataopslag.\\

Beginnende in de jaren 2000 groeide in de IT-wereld dan ook het besef dat de rigide "one size fits all"-aanpak van RDBMS voor vele moderne toepassingen achterhaald dreigde te geraken. Met grote opkomende spelers uit de <eb-industrie zoals Google, Amazon en Facebook aan het roer leidde dit tot de opkomst van de NoSQL-beweging. NoSQL staat, in tegenstelling tot wat de naam doet vermoeden, voor "Not only SQL" en omvat een waaier van uiteenlopende alternatieve gegevensopslagsystemen die elk in bepaalde specifieke opzichten meerwaarde trachten te bieden ten opzichte van het klassieke relationele systemen. In tegenstelling tot de 'Zwitsers zakmes'-aanpak van RDMBS, leggen ze zich toe op zeer gespecialiseerde toepassingsdomeinen en proberen daarin relationele systemen te overtreffen. Vaak betekent dit dat NoSQL systemen vele voor hun doel onnodig geachte features van SQL systemen achterwege laten, of afzwakken. Een goed voorbeeld hiervan zijn de ACID-eigenschappen uit het relationele model die in vele NoSQL-systemen gereduceerd zijn tot zogenaamde BASE-eigenschappen (op het verschil tussen beide komt onderstaande sectie nog uitgebreid terug).

\section{Begrippen}

Deze sectie belicht de belangrijkste technische begrippen in verband met NoSQL-databanken en contrasteert waar nodig met gelijkaardige concepten in het klassieke traditionele model.

\paragraph{Consistentie}

Consistentie van database-transacties betekent dat transacties de databank in een consistente staat achterlaten: alle data die een applicatie kan zien, is een consistente snapshot van de databank \cite{ports2010transactional}. Traditionele RDBMS bieden vaak transacties met de zogenaamde ACID-eigenschappen \cite{haerder1983principles}:
\begin{itemize}
\item \textbf{Atomicity:} Elke transactie gebeurt ofwel volledig, ofwel helemaal niet.
\item \textbf{Consistency:} Elke transactie laat de databank in consistente staat achter.
\item \textbf{Isolation:} Elke transactie verloopt volledig ge\"isoleerd van elke andere transactie en be\"invloedt deze dus op geen enkele manier.
\item \textbf{Durability:} Eens voltrokken, blijft elke transactie duurzaam bewaard in de databank, ook in het geval van stroomonderbrekingen, crashes of fouten.
\end{itemize}

Zoals Eric Brewer stelde in zijn bekende CAP-theorema \cite{brewer2000towards}, is het in een gedistribueerd systeem niet eenvoudig zowel consistentie, availability als tolerantie voor partities te bereiken en zijn 2 van deze 3 eigenschappen het hoogst haalbare\footnote{Brewer kwam hier zelf 12 jaar later op terug, stellende dat mits goede omgang met partities het toch mogelijk is een trade-off van alle drie te bereiken \cite{brewer2012cap}.}.
Ook in NoSQL-systemen, die vaak gedistribueerd van aard zijn, is het garanderen van consistentie geen triviale opgave. Afhankelijk van de gehanteerde schrijfstrategie is het mogelijk dat verschillende knopen in het cluster verschillende versies van data zien, als updates nog niet in het volledige cluster doorgekomen zijn. Daarom is er het onderscheid tussen strikte en uiteindelijke ("\textit{eventual}") consistentie: strikte consistentie is de gekende vorm waarin updates onmiddellijk zichtbaar zijn op alle nodes in het cluster, en dus ook naar bovenliggende applicaties toe. In het geval van uiteindelijke consistentie garandeert het systeem enkel dat na verloop van tijd alle nodes in het cluster dezelfde, up-to-date versie van de data zullen zien. NoSQL-systemen bieden dan ook vaak de BASE-eigenschappen, een zwakkere versie van de ACID-garanties:
\begin{itemize}
\item \textbf{Basically available:} Het systeem is onder quasi alle omstandigheden beschikbaar.
\item \textbf{Soft state:} Het systeem verkeert niet altijd in een consistente staat
\item \textbf{Eventually consistent:} Na verloop van tijd zal het systeem in een gekende staat verkeren.
\end{itemize}

Vele NoSQL-systemen stellen de gebruiker echter niet voor een voldongen feit bij de keuze tussen strikte en uiteindelijke consistentie: dankzij zogenaamde \textit{quora} kan de gebruiker zelf configureren welke consistentie het systeem levert. Door lees- en schrijfquora in te stellen, kan de gebruiker tunen hoeveel replica's respectievelijk moeten returnen bij een schrijfopdracht, en het welslagen van een schrijfopdracht moeten bevestigen. Op deze manier kan de gebruiker zelf een trade-off maken tussen snel lezen, schrijven en de behaalde consistentie. Bovendien zal de gebruiker, wanneer de som van het lees- en schrijfquorum groter is dan de replicatiefactor van het cluster, steeds de meest recente versie van gegevens zien, wat hetzelfde betekent als onmiddellijke, strikte consistentie.

\section{NoSQL-klassen}

NoSQL databanken zijn er in verschillende soorten en kunnen op basis van hun datamodel in een aantal categorie\"en onderverdeeld worden:

\begin{itemize}
\item \textbf{Key-Value stores} Deze zijn vergelijkbaar met dictionaries en mappen unieke keys op values. Deze values zijn voor de databank volledig betekenisloze byte-arrays en de enige manier om ze op te vragen, is via de bijhoren key. Voor zeer eenvoudige toepassingen resulteert dit in hoge lees- en schrijf-throughput, maar meer geavanceerde features zoals indexing, queries, en het modelleren van relaties binnen de data zijn hierdoor niet mogelijk\cite{hecht2011nosql}\cite{grolinger2013data}.

\item \textbf{Columnar stores} Deze zijn gebaseerd op het datamodel dat Google's BigTable heeft ge\"introduceerd en slaan data op in een spaarse, gedistribueerde, persistente en multidimensionele gesorteerde map\cite{chang2008bigtable}. In het geval van BigTable zijn dit drie dimensies: row key, column key en een timestamp. Omdat ook hier het systeem de opgeslagen data niet interpreteert, is het modelleren van relaties niet op een effici\"ente manier mogelijk. Dit wordt overgelaten aan de bovenliggende applicatie\cite{hecht2011nosql}.

\item \textbf{Document stores} Deze bewaren data als key-value paren en encapsuleren deze in documenten. Values kunnen van een brede waaier aan types zijn, zoals geneste documenten, lijsten of scalars. De namen van attributen kunnen dynamisch gespecifieerd worden tijdens runtime en moeten geen vooraf vastgelegd schema volgen\cite{cattell2011scalable}. Dit is geschikt voor het modelleren van ingewikkelde datastructuren. Vele document stores gebruiken het JSON-bestandsformaa (of een daarvan afgeleide vorm). In tegenstelling tot columnar stores, zijn de waarden in documenten niet betekenisloos voor het document store systeem en het is dus mogelijk hier indexen op te defini\"eren en queries op uit te voeren\cite{hecht2011nosql}.

\item \textbf{Graph databases} Zoals de naam doet vermoeden, stammen deze uit grafentheorie en maken ze gebruik van grafen als datamodel. Ze zijn uitermate geschikt om sterk verweven data te beheren, van bronnen zoals sociale netwerken of location based services. Hiervoor doen ze beroep op effici\"ente mechanismen om grafen te doorlopen waar andere systemen kostelijke operaties als recursieve joins gebruiken\cite{hecht2011nosql}.

\end{itemize}

De term NewSQL slaat op een verzameling systemen die ambi\"eren het klassieke relationele datamodel te combineren met de schaalbaarheid, distributie en fouttolerantie van NoSQL systemn. Hoewel ze allen de gebruiker het relationele model en SQL-achtige query mogelijkheden bieden, verschillen NewSQL stores onderling grondig, afhankelijk van de onderliggende architectuur. Zo zijn er onder meer systemen die gebouwd zijn bovenop bestaande NoSQL databanken, en andere die alle data in main memory opslaan.\cite{grolinger2013data}


