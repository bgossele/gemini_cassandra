\chapter{Achtergrond databanken}
\label{cha2}

\section{NoSQL}

NoSQL databanken zijn er in verschillende soorten en kunnen op basis van hun datamodel in een aantal categorie\"en onderverdeeld worden:

\begin{itemize}
\item[Key-Value stores] Deze zijn vergelijkbaar met dictionaries en mappen unieke keys op values. Deze values zijn voor de databank volledig betekenisloze byte-arrays en de enige manier om ze op te vragen, is via de bijhoren key. Voor zeer eenvoudige toepassingen resulteert dit in hoge lees- en schrijf-throughput, maar meer geavanceerde features zoals indexing, queries, en het modelleren van relaties binnen de data zijn hierdoor niet mogelijk\cite{hecht2011nosql}\cite{grolinger2013data}.

\item[Columnar stores] Deze zijn gebaseerd op het datamodel dat Google's BigTable heeft ge\"introduceerd en slaan data op in een spaarse, gedistribueerde, persistente en multidimensionele gesorteerde map\cite{chang2008bigtable}. In het geval van BigTable zijn dit drie dimensies: row key, column key en een timestamp. Omdat ook hier het systeem de opgeslagen data niet interpreteert, is het modelleren van relaties niet po een effici\"ente manier mogelijk. Dit wordt overgelaten aan de bovenliggende applicatie\cite{hecht2011nosql}.

\item[Document stores] Deze bewaren data als key-value paren en encapsuleren deze in documenten. Values kunnen van een brede waaier aan types zijn, zoals geneste documenten, lijsten of scalars. De namen van attributen kunnen dynamisch gespecifieerd worden tijdens runtime en moeten geen vooraf vastgelegd schema volgen\cite{cattell2011scalable}. Dit is geschikt voor het modelleren van ingewikkelde datastructuren. Vele document stores gebruiken het JSON-bestandsformaa (of een daarvan afgeleide vorm). In tegenstelling tot columnar stores, zijn de waarden in documenten niet betekenisloos voor het document store systeem en het is dus mogelijk hier indexen op te defini\"eren en queries op uit te voeren\cite{hecht2011nosql}.

\item[Graph databases] Zoals de naam doet vermoeden, stammen deze uit grafentheorie en maken ze gebruik van grafen als datamodel. Ze zijn uitermate geschikt om sterk verweven data te beheren, van bronnen zoals sociale netwerken of location based services. Hiervoor doen ze beroep op effici\"ente mechanismen om grafen te doorlopen waar andere systemen kostelijke operaties als recursieve joins gebruiken\cite{hecht2011nosql}.

\end{itemize}

De term NewSQL slaat op een verzameling systemen die ambi\"eren het klassieke relationele datamodel te combineren met de schaalbaarheid, distributie en fouttolerantie van NoSQL systemn. Hoewel ze allen de gebruiker het relationele model en SQL-achtige query mogelijkheden bieden, verschillen NewSQL stores onderling grondig, afhankelijk van de onderliggende architectuur. Zo zijn er onder meer systemen die gebouwd zijn bovenop bestaande NoSQL databanken, en andere die alle data in main memory opslaan.\cite{grolinger2013data}


