\chapter{NoSQL: vergelijkende studie}
\label{nosql_survey}

\section{Methodologie}

Omwille van het enorme aanbod aan NoSQL en NewSQL systemen was een exhaustieve studie niet haalbaar. Deze vergelijkende studie behandelt de populairste datastores in een aantal relevante categorie\"en, namelijk document stores, columnar stores en NewSQL stores. Key-value stores en graph databases komen niet aan bod, aangezien hun datamodellen niet geschikt zijn voor de toepassing in kwestie. De uiteindelijke selectie werd gemaakt volgens criteria vergelijkbaar met die in \cite{grolinger2013data}, met de ranking van DB-Engine Ranking \cite{db_engine_rank} als maat voor de populariteit.\\

Deze ranglijst tracht populariteit te meten op basis van enkele parameters, zoals aantal vermeldingen op websites, algemene interesses volgens Google Trends, frequentie van technische discussies op fora zoals StackOverflow, vacatures i.v.m. de technologie en vermeldingen in professionele profielen op sites zoals LinkedIn. De resulterende selectie bestaat uit de document stores MongoDB en CouchBase Server, wide columnar stores Cassandra en HBase en NewSQL database VoltDB. Er bestaat al een uitbreiding van de DNA sequencing pijplijn die het ExaScience Life Lab gebruikt om MongoDB  databanken als in- en/of uitvoer te gebruiken voor de pijplijn. Dit maakt MongoDB uiteraard nog relevanter. Ten laatste werd ook NewSQL query engine Cloudera Impala in de studie betrokken wegens expliciete interesse van onderzoekers in het eerder vernoemde lab. Deze 6 systemen vergeleek ik 
%TODO ik?
vervolgens op een aantal voor high performance computing relevante eigenschappen zoals indexeringsmechanismen, interfaces naar de gebruiker en API's, distributiestrategie, concurrency controle en consistentiemodel.

\subsection{Document stores}

\paragraph{MongoDB}

MongoDB slaat gegevens op in BSON (binary JSON) documenten. Het systeem biedt krachtige ondersteuning voor indices, vooral door de mogelijkheid om secundaire indices van een brede waaier van types te defini\"eren op alle attributen, zoals in het relationele model. Deze indices zijn gebouwd op B-trees\cite{mongodb_indexes}. Om denormalizatie te bevorderen, kunnen documenten geneste documenten en arrays bevatten. Zo zijn joins ook overbodig in de query taal.\\
De bestandsgrootte is beperkt tot 16 MB, om te voorkomen dat \'e\'en enkel document buitensporig veel RAM of bandbreedte opeist. Om grotere bestanden te bewaren, kan het ingebouwde GridFS (dat integenstelling tot wat de naam doet vermoeden, geen volwaardig file system is) automatisch bestanden opsplitsen in kleinere delen en deze delen als aparte documenten bewaren, zonder dat de gebruiker zich hierom moet bekommeren.\\
MongoDB biedt API's in zeer vele programmeertalen en de functionaliteit om het equivalent van SQL \texttt{WHERE}-clausules te defini\"eren als javascript uitdrukkingen. MongoDB vertaalt deze vervolgens naar een eigen, interne en afgeschermde query taal \cite{grolinger2013data}. De query optimizer van MongoDB verwerkt queries en kiest voor elke query een zo effici\"ent mogelijk uitvoeringsplan gegeven de beschikbare indices. Deze plannen worden gecached als er meerdere goede alternatieven zijn en kunnen geherevalueerd worden naarmate de gegevensset in de databank evolueert.\\
Qua consistentie laat MongoDB de keuze tussen uiteindelijke en strikte consistentie. Strikte consistentie is mogelijk door ofwel enkel te lezen van de master node (die de meest up-to-date versie van de data heeft) of na schrijfopdrachten te wachten tot alle replica's bevestigd hebben alvorens verder te gaan. De eerste optie introduceert een bottleneck bij het lezen van data, de tweede verhoogt de latentie van schrijfopdrachten.\\
MongoDB repliceert data asynchroon en partitioneert in ranges: nodes zijn verantwoordelijk voor ranges van keys. Dit zorgt voor snelle range queries, maar kan hotspots en load-balancingproblemen veroorzaken. Dankzij een master-slave struktuur kan MongoDB updates gemakkelijk naar de juiste replica's doorverwijzen.
Op het gebied van concurrency controle biedt MongoDB atomiciteit binnen documenten en reader-writer locks. 
%TODO beter woord dan locken
Bij schrijfopdrachten de databank locken heeft een zware impact op de performantie in scenario's waar veel geschreven moet worden.\\\\
Kortom, MongoDB bewaart BSON-bestanden op een zeer toegankelijke manier met flexibele query- en indexeringsmechanismes. De concurrency- en consistentiemodellen daarentegen vertonen enkele nadelen.

\paragraph{Couchbase Server}

Couchbase, het resultaat van de fusie tussen CouchDB en Membase, slaat gegevens op in JSON documenten. Het hanteert het memcached protocol om een gedistribueerde cache en is bedoeld voor zeer interactieve toepassingen met hoge vereisten op gebied van latentie \cite{grolinger2013data}\cite{couchbase_about}.
De JSON documenten kunnen genest zijn en kunnen doorzocht worden met een uitgebreide, SQL-achtige taal, N1QL (op het moment van schrijven is dit wel nog steeds een developer preview, uitgebracht in januari 2015) \cite{couchbase_n1ql}.
Net als MongoDB kunnen primaire en secundaire indices gedefinieerd worden en zijn deze gestoeld op B-trees \cite{couchbase_index}.\\
Binnen \'e\'en cluster zijn transacties strikt consistent, maar tussen meerder clusters slechts uiteindelijk consistent.\\
CouchBase biedt gebruikers de keuze tussen optimistische (m.b.v. compare-and-swap) en pessimistische (m.b.v. 'finegrained locking') concurreny controle.\\\\
Dankzij zijn flexibele datamodel, caching en concurrency controle, past CouchBase goed voor toepassingen die snelle en intensieve interactieve vergen tussen gebruiker en data.

\subsection{Columnar stores}

\paragraph{Cassandra}

Cassandra werd oorspronkelijk ontwikkeld voor intern gebruik bij Facebook maar is later als Apache opensourceproject publiekelijk beschikbaar gemaakt. Het combineert het datamodel van Google's BigTable systeem met de architectuur en distributiestrategie van Amazons DynamoDB. Het is gericht op flexibele, quasipermanent beschikbare opslag van zeer grote datasets op goedkope standaardhardware, met daarenboven hoge througput voor schrijfopdrachten zonder effici\"entie bij leesopdrachten op te offeren \cite{borthakur2011apache}.\\
Sinds zijn ontstaan is Cassandra wel op enkele vlakken afgeweken van het BigTable-model\cite{cassandra_there&now}, in die zin dat het nu tabellen en samengestelde kolommen
%TODO betere vertaling voor composite columns?
biedt, evenals een eigen query taal, CQL \cite{cassandra_CQL}. CQL vertoont op het gebied van syntax en functionaliteit sterke gelijkenissen met SQL, maar is toch sterk beperkt. Zo biedt het bijvoorbeeld geen \texttt{JOIN}-clausule, en zijn \texttt{WHERE}-clausules aan sterke voorwaarden onderhevig. Cassandra moedigt het samenbewaren van gegevens die samen opgevraagd worden sterk aan en ondersteunt denormalizatie met features zoals collection types.\\
Cassandra heeft indexeringsmechanismes and implementeert deze met log-structured merge trees, met hogere schrijfthroughput als gevolg.
%TODO hier uitleggen of eerder meer nr boven?
Net als BigTable biedt Cassandra ook Bloom filters.\\
Om lineair in het aantal ingeschakelde nodes te kunnen schalen naar zeer grote datasets, opereert Cassandra op een volledig hi\"erarchieloze wijze. Vanuit het perspectief van het CAP-theorema, spitst Cassandra zich toe op availability en partition tolerance, ten koste van onmiddellijke consistentie. Het consistentieniveau kan wel per query door de gebruiker bepaald worden, zoals later verduidelijkt wordt. Hoge beschikbaarheid en tolerantie voor fouten bereikt Cassandra door asynchroon data te repliceren over verschillende nodes, met consistent hashing en virtuele nodes om frequent komen-en-gaan en incrementeel toevoegen van nodes op te vangen. Het aantal replica's kan de gebruiker zelf kiezen. Bovendien voorziet Cassandra ook interdatacenterreplicatie, om zelfs het falen van volledige datacenters op te vangen\cite{decandia2007dynamo} \cite{lakshman2010cassandra} \cite{cassandra_then&now}.\\
Bij lees- en schrijfopdrachten kan de gebruiker een quorum specifi\"eren. Hoewel Cassandra met uiteindelijke consistentie voor het oog ontworpen werd, is onmiddellijke consistentie mits een juiste keuze van de quorum dus ook een optie.\\
Op het gebied van concurrency controle garandeert Cassandra atomiciteit binnen rijen en serializeerbare \textit{lightweight transactions}, eigenlijk compare-and-set functionaliteit, voor grotere operaties.
\\\\
Samengevat biedt Cassandra redelijk flexibele datamodellering met (licht beperkte) query- en indexeringsmechanismen via de CQL-interface. Het sterkste punt is echter dat Cassandra incrementeel schaalt naar enorme datasets, dankzij uitvoerige replicatie- en foutverwerkingsfeatures.

