\chapter{GEMINI overzicht}
\label{cha3}

Zoals eerder vermeld is GEMINI een applicatie voor de flexibele analyse van genoomdata van populaties van menselijke individu\"en. Deze sectie gaat dieper in op de belangrijkste features en het onderliggende datamodel van GEMINI in zijn oorspronkelijke vorm.

\section{Datamodel}

GEMINI importeert genetische variants en genotypes van alle gesampelde individu\"en (ook 'samples') vanuit een VCF file in een relationele database.
Daarnaast kan extra informatie over de samples, zoals geslacht, phenotype en onderlinge verwantschappen, meegegeven worden in een PED-file (van pedigree) om latere analyse te vergemakkelijken.\\

Elke variant in een input VCF file wordt uitvoerig geannoteerd na automatische vergelijking met bestaande of door de gebruiker gedefinieerde genoom-annotatiebestanden. De geannoteerde variants vormen de rijen van de hoofdtabel van de database, de \texttt{variants}-tabel. Deze tabel bevat ook voor elke variant informatie over elke sample, zoals diens genotype, de kwaliteit en diepte van de meting voor de variant in kwestie. In de SQLite-versie van GEMINI wordt dit opgeslagen als een gecomprimeerde array per variant, 1 voor elke sample-eigenschap: zo is er een \texttt{gt\_type}-kolom met arrays met de genotypes, en een \texttt{gt\_depth}-kolom met arrays met de diepte van de meting van elke sample voor elke variant. Samen met de \texttt{samples}-tabel, die voor elke sample zaken als het geslacht, phenotype en familierelaties bijhoudt, ligt de \texttt{variants}-tabel aan de basis van de uitgebreide query-mogelijkheden die GEMINI biedt.\\

Daarnaast zijn er nog tabellen zoals de \texttt{variant\_impacts}- en \texttt{gene\_detailed}-tabellen die respectievelijk extra informatie over de variants en het menselijk genoom bevatten. Deze informatie komt in het eerste geval eveneens uit de annotatiebestanden, en in het tweede uit tekstbestanden met referentie-informatie over het menselijk genoom die GEMINI, indien gewenst mee inlaadt.\\
Ten laatste zijn er nog enkele kleine tabellen met meta-informatie, zoals de \texttt{resources}-tabel die de gebruikte annotatie-files bevat, en de \texttt{version}-tabel die bevat doorn welke versie van GEMINI de data ingeladen is.\\\\

Een belangrijke troef van GEMINI en z'n datamodel is de flexibiliteit die het laat naar de gebruiker. Zo kan de gebruiker zelfgedefinieerde annotatie-files gebruiken, en zelf kolommen toevoegen aan de PED-files met informatie over de samples. Deze extra informatie zal GEMINI automatisch in de resp. \texttt{variants}- en \texttt{samples}-tabel opnemen en kan de gebruiker later ook doorzoeken.

\begin{figure}[h]
\includegraphics[scale=0.8]{gemini_schema}
\caption{Een overzicht van GEMINI's schema}
\label{gemini_schema_pic}
\end{figure}

\section{Inladen}
%TODO ref nr IPython dink
Het inladen van de data uit VCF-bestanden is een computationeel intensieve operatie, enerzijds omwille van de enorme grootte van deze bestanden, en anderzijds omdat in deze fase ook alle variants geannoteerd moeten worden. Om het proces te versnellen, biedt GEMINI de mogelijkheid het werk te paralleliseren door het VCF-bestand de comprimeren, op het bestaande bestand een index te defini\"eren en zo het werk te verdelen. Dit kan simpelweg over meerdere processoren binnen 1 computer zijn, maar via de IPython-interface ook over volledige clusters van computers.\\


